
\chapter{An Overview and Brief History of Feedback Control}

\section{Review Questions}

\begin{problem}
   What are the main components of a feedback control system?
\end{problem}

\begin{figure}
   \begin{center}
      %\includegraphics{control}
      \begin{tikzpicture}
      \tikzset{vertex/.style = {shape=circle,draw,minimum size=10ex}}
      \tikzset{edge/.style = {->,> = latex'}}
      \fill[green!40!white] (-1.25,-6.25) rectangle(1.25,1.25);
      \node[green!40!black,left] at (-1.25,-2.5) {logical controller};
      \fill[blue!40!white] (-1.25,-11.25) rectangle(1.25,-6.25);
      \node[blue!40!black,left] at (-1.25,-8.75) {plant};
      % vertices
      \node[vertex] (a) at  (0,  0) {input filter};
      \node[vertex] (b) at  (0, -2.5) {$\Sigma$};
      \node[above,outer sep=28pt] at (0.2, -2.5) {$+$};
      \node[below right,outer sep=12pt] at (0.4, -2.5) {$-$};
      \node[vertex] (c) at  (0, -5) {controller};
      \node[vertex] (d) at  (0, -7.5) {actuator};
      \node[vertex] (e) at  (0,-10) {process};
      \node[vertex] (f) at  (2.5,-10) {sensor};
      %edges
      \draw[edge] (a) to (b);
      \draw[edge] (b) to (c);
      \draw[edge] (c) to (d);
      \draw[edge] (d) to (e);
      \draw[edge] (e) to (f);
      \draw[edge] (f) to[bend right] (b);
      \end{tikzpicture}
      \caption{%
         Components of a feedback-control system. The comparator is represented
         by the symbol `$\Sigma$', the symbol for summation, because a
         difference is a special kind of sum. The comparator takes a difference
         between two quantities by adding one to the negative of the other. In
         this case, the comparator adds the value from the input filter to the
         negative of the output of the sensor. In some cases, the input filter
         and the comparator may be integrated as part of the controller.%
      }
      \label{fig:control}
   \end{center}
\end{figure}

\begin{solution}
   A feedback control system has
   \begin{enumerate}
      \item a \emph{sensor} that measures the output that is to be controlled,
      \item an \emph{input filter} that provides a reference value against
         which the sensor's output is compared,
      \item a \emph{comparator} that computes the difference between the output
         of the sensor and the output of the input filter,
      \item a \emph{controller} that translates the output of the comparator
         into a control signal,
      \item a \emph{plant}, which consists of
         \begin{enumerate}
            \item an \emph{actuator} that receives the control signal and
               causes a physical change in the system and
            \item a \emph{process} that can be disturbed by environmental
               forces but is primarily influenced by the actuator to produce an
               output measured by the sensor.
         \end{enumerate}
   \end{enumerate}
   See Figure~\ref{fig:control}. Because the input filter and the comparator
   may sometimes be integrated into the controller, the main components of a
   feedback-control system are the controller, the actuator, the process, and
   the sensor. The most compact list of parts in a feedback-control system is
   therefore as follows: controller, plant, and sensor.
\end{solution}

\begin{problem}
   What is the purpose of the sensor?
\end{problem}

\begin{solution}
   The designer of the control system intends that a particular quantity $Q$ be
   controlled. The purpose of the sensor is to measure $Q$ or, when that be
   impossible, to measure a related quantity $Q'$ from which $Q$ may be
   estimated.
\end{solution}

\begin{problem}
   What are three properties of a good sensor?
\end{problem}

\begin{solution}
   A good sensor
   \begin{enumerate}
      \item introduces sufficiently little noise into its measurement,
      \item produces a measurement that in the mean (apart from noise) and to a
         sufficiently good approximation is linearly related to the quantity
         that the system is intended to control, and
      \item is reliable.
   \end{enumerate}
   A good sensor allows the controller to make a right decision about what the
   actuator should do.
\end{solution}

\begin{problem}
   What is the purpose of the actuator?
\end{problem}

\begin{solution}
   The designer of the control system intends that a particular quantity $Q$ be
   controlled. The purpose of the actuator is to change the value of $Q$.
\end{solution}

\begin{problem}
   What are three properties of a good actuator?
\end{problem}

\begin{solution}
   A good actuator
   \begin{enumerate}
      \item has the power to change the system's output quantity $Q$,
         regardless of disturbance to the system,
      \item can change $Q$ sufficiently quickly, and
      \item is reliable.
   \end{enumerate}
   The actuator and the process together make up the plant. So the design of a
   good actuator is part of the design of a good plant.
\end{solution}

\begin{problem}
   What is the purpose of the controller? Give the input(s) and the output(s)
   of the controller.
\end{problem}

\begin{solution}
   The designer of the control system intends that a particular quantity $Q$ be
   controlled. The purpose of the controller is to decide how the actuator
   should behave. The input to the controller is the difference between the
   desired value of $Q$ and the measured value of $Q$.  The output of the
   controller is a signal that determines the behavior of the actuator.
\end{solution}

\begin{problem}
   What physical variable(s) of a process can be directly measured by a
   Hall-effect sensor?
\end{problem}

\begin{solution}
   A Hall-effect sensor can measure either a current or a magnetic field. If
   the current through the sensor be constant, then the Hall voltage indicates
   the strength of that component of the magnetic field perpendicular to the
   plane of the sensor. If the magnetic field be constant, then the Hall
   voltage indicates the magnitude of the current through the sensor.
\end{solution}

\begin{problem}
   What physical variable is measured by a tachometer?
\end{problem}

\begin{solution}
   A tachometer measures angular speed.
\end{solution}

\begin{problem}
   What are three different techniques for measuring temperature?
\end{problem}

\begin{solution}
   Temperature might be measured
   \begin{enumerate}
      \item by way of the voltage across a resistor (a thermistor), whose
         resistance changes with temperature, so that the resistor serves as
         the load for a constant-current source,
      \item by way of a bimetallic strip coiled into spring whose point of
         mechanical equilibrium varies with temperature, or
      \item by way of a pair of photometers, each of which measures the
         radiation intensity at a different wavelength near the peak of the
         ambient blackbody radiation field.
   \end{enumerate}
   Each of these has many variations.
\end{solution}

\begin{problem}
   Why does the typical sensor have an electrical output, regardless of the
   physical nature of the measured variable?
\end{problem}

\begin{solution}
   An electrical signal can quickly be communicated over a long distance to the
   comparator; further, the difference between electrical signals, one from the
   input filter and the other from the sensor, may easily be obtained, whether
   by analog or by digital means.
\end{solution}

