
\chapter{An Overview and Brief History of Feedback Control}

Chapter~1 establishes the basic terminology used to describe a control system.
Both historic and current examples are given.

\section{Review Questions}

\subsection{Components of Control System}

\subsubsection{Problem}

What are the main components of a feedback control system?

\subsubsection{Solution}

\begin{figure}
   \begin{center}
      %\includegraphics{control}
      \begin{tikzpicture}
      \tikzset{vertex/.style = {shape=circle,draw,minimum size=10ex}}
      \tikzset{edge/.style = {->,> = latex'}}
      \fill[green!40!white] (-1.25,-6.25) rectangle(1.25,1.25);
      \node[green!40!black,left] at (-1.25,-2.5) {logical controller};
      \fill[blue!40!white] (-1.25,-11.25) rectangle(1.25,-6.25);
      \node[blue!40!black,left] at (-1.25,-8.75) {plant};
      % vertices
      \node[vertex] (a) at  (0,  0) {input filter};
      \node[vertex] (b) at  (0, -2.5) {$\Sigma$};
      \node[above,outer sep=28pt] at (0.2, -2.5) {$+$};
      \node[below right,outer sep=12pt] at (0.4, -2.5) {$-$};
      \node[vertex] (c) at  (0, -5) {controller};
      \node[vertex] (d) at  (0, -7.5) {actuator};
      \node[vertex] (e) at  (0,-10) {process};
      \node[vertex] (f) at  (2.5,-10) {sensor};
      %edges
      \draw[edge] (a) to (b);
      \draw[edge] (b) to (c);
      \draw[edge] (c) to (d);
      \draw[edge] (d) to (e);
      \draw[edge] (e) to (f);
      \draw[edge] (f) to[bend right] (b);
      \end{tikzpicture}
      \caption{%
         Components of a feedback-control system. The comparator is represented
         by the symbol `$\Sigma$', the symbol for summation, because a
         difference is a special kind of sum. The comparator takes a difference
         between two quantities by adding one to the negative of the other. In
         this case, the comparator adds the value from the input filter to the
         negative of the output of the sensor. In some cases, the input filter
         and the comparator may be integrated as part of the controller.%
      }
      \label{fig:control}
   \end{center}
\end{figure}

A feedback control system has
\begin{enumerate}
   \item a \emph{sensor} that measures the output that is to be controlled,
   \item an \emph{input filter} that provides a reference value against which
      the sensor's output is compared,
   \item a \emph{comparator} that computes the difference between the output of
      the sensor and the output of the input filter,
   \item a \emph{controller} that translates the output of the comparator into
      a control signal,
   \item a \emph{plant}, which consists of
      \begin{enumerate}
         \item an \emph{actuator} that receives the control signal and causes a
            physical change in the system and
         \item a \emph{process} that can be disturbed by environmental forces
            but is primarily influenced by the actuator to produce an output
            measured by the sensor.
      \end{enumerate}
\end{enumerate}
See Figure~\ref{fig:control}. Because the input filter and the comparator may
sometimes be integrated into the controller, the main components of a
feedback-control system are the controller, the actuator, the process, and the
sensor. The most compact list of parts in a feedback-control system is
therefore as follows: controller, plant, and sensor.

\subsection{Purpose of Sensor}

\subsubsection{Problem}

What is the purpose of the sensor?

\subsubsection{Solution}

The designer of the control system intends that a particular quantity $Q$ be
controlled. The purpose of the sensor is to measure $Q$ or, when that be
impossible, to measure a related quantity $Q'$ from which $Q$ may be estimated.

\subsection{Properties of Sensor}

\subsubsection{Problem}

What are three properties of a good sensor?

\subsubsection{Solution}

A good sensor
\begin{enumerate}
   \item introduces sufficiently little noise into its measurement,
   \item produces a measurement that in the mean (apart from noise) and to a
      sufficiently good approximation is linearly related to the quantity that
      the system is intended to control, and
   \item is reliable.
\end{enumerate}
A good sensor allows the controller to make a right decision about what the
actuator should do.

\subsection{Purpose of Actuator}

\subsubsection{Problem}

What is the purpose of the actuator?

\subsubsection{Solution}

The designer of the control system intends that a particular quantity $Q$ be
controlled. The purpose of the actuator is to change the value of $Q$.

\subsection{Properties of Actuator}

\subsubsection{Problem}

What are three properties of a good actuator?

\subsubsection{Solution}

A good actuator
\begin{enumerate}
   \item has the power to change the system's output quantity $Q$, regardless
      of disturbance to the system,
   \item can change $Q$ sufficiently quickly, and
   \item is reliable.
\end{enumerate}
The actuator and the process together make up the plant. So the design of a
good actuator is part of the design of a good plant.

\subsection{Purpose of Controller}

\subsubsection{Problem}

What is the purpose of the controller? Give the input(s) and the output(s) of
the controller.

\subsubsection{Solution}

The designer of the control system intends that a particular quantity $Q$ be
controlled. The purpose of the controller is to decide how the actuator should
behave. The input to the controller is the difference between the desired value
of $Q$ and the measured value of $Q$.  The output of the controller is a signal
that determines the behavior of the actuator.

\subsection{Hall-Effect Sensor}

\subsubsection{Problem}

What physical variable(s) of a process can be directly measured by a
Hall-effect sensor?

\subsubsection{Solution}

A Hall-effect sensor can measure either a current or a magnetic field. If the
current through the sensor be constant, then the Hall voltage indicates the
strength of that component of the magnetic field perpendicular to the plane of
the sensor. If the magnetic field be constant, then the Hall voltage indicates
the magnitude of the current through the sensor.

\subsection{Tachometer}

\subsubsection{Problem}

What physical variable is measured by a tachometer?

\subsubsection{Solution}

A tachometer measures angular speed.

\subsection{Measurement of Temperature}

\subsubsection{Problem}

What are three different techniques for measuring temperature?

\subsubsection{Solution}

Temperature might be measured
\begin{enumerate}
   \item by way of the voltage across a resistor (a thermistor), whose
      resistance changes with temperature, so that the resistor serves as the
      load for a constant-current source,
   \item by way of a bimetallic strip coiled into spring whose point of
      mechanical equilibrium varies with temperature, or
   \item by way of a pair of photometers, each of which measures the radiation
      intensity at a different wavelength near the peak of the ambient
      blackbody radiation field.
\end{enumerate}
Each of these has many variations.

\subsection{Sensor with Electrical Output}

\subsubsection{Problem}

Why does the typical sensor have an electrical output, regardless of the
physical nature of the measured variable?

\subsubsection{Solution}

An electrical signal can quickly be communicated over a long distance to the
comparator; further, the difference between electrical signals, one from the
input filter and the other from the sensor, may easily be obtained, whether by
analog or by digital means.

\section{Problems}

\subsection{Block Diagrams}

\subsubsection{Problem}

Draw a component block diagram for each of the following feedback-control
systems.
\begin{enumerate}[(a)] \itemsep1pt \parskip0pt \parsep0pt
   \item The manual steering system of an automobile.
   \item Drebbel's incubator.
   \item The water level controlled by a float and valve.
   \item Watt's steam engine with fly-ball governor.
\end{enumerate}
In each case, indicate the location of each element listed below, and give the
units associated with each signal.
\begin{itemize} \itemsep1pt \parskip0pt \parsep0pt
   \item controller and its output signal
   \item actuator and its output signal
   \item process and its output signal
   \item sensor
   \item reference signal
   \item error signal
\end{itemize}
Notice that in each of a number of cases, the same physical device may perform
more than one of these functions.

\subsubsection{Solution}

There is a separate figure for each block diagram.
\begin{figure}
   \begin{center}
      \begin{small}
         \input{pr01-01-a}
      \end{small}
      \caption{Block diagram for manual steering in an automobile.}
      \label{fig:manual-steering}
   \end{center}
\end{figure}
\begin{figure}
   \begin{center}
      \begin{small}
         \input{pr01-01-b}
      \end{small}
      \caption{Block diagram for Drebbel's incubator.}
      \label{fig:incubator}
   \end{center}
\end{figure}
\begin{enumerate}[(a)]
   \item For the manual steering system, see Figure~\ref{fig:manual-steering}.
      \begin{itemize}
         \item Functioning as the controller, the mind uses the steering error
            to tell muscles to contract. The control signal is an increase or
            decrease in activity along motor neurons to the arms and hands.
         \item The actuator consists of the driver's hands on the steering
            wheel, along with the mechanism that conveys the direction angle to
            the front wheels. The signal from the actuator is the angle of the
            front wheels on the road; this signal could be expressed in degrees
            or radians.
         \item The process is the the automobile as it moves over the road; the
            relevant signal output from the process is the radiation field
            sensible through the windshield of the automobile. The relevant
            dimension in the output signal is the angle at the driver's head
            between the direction of travel and the light coming from the
            center-line of the road. This angle might be measured in degrees or
            radians.
         \item The sensor consists of the driver's eyes.
         \item The imagination is an input filter that converts the idea of the
            automobile's best state into an ideal image of the road as the
            driver ought to see it.  This is the reference signal. The relevant
            dimension in the reference signal is the angle at the driver's head
            between the direction of travel and the center-line of the road.
         \item This imagined reference image is compared with the observed
            image that comes from the eyes, which serve as the sensor.  The
            mind, which serves as comparator, compares the imagined reference
            image with the observed image and derives an angular steering
            error, which might be measured in degrees, but the error signal
            might be a neural state.
      \end{itemize}
   \item For Drebbel's incubator, see Figure~\ref{fig:incubator}.
      \begin{itemize}
         \item Functioning as the controller, the flew, in response to the
            offset of the float from its reference height, shifts away from the
            half-open position. The offset is the error signal.  The control
            signal is the angle of the flew. The angle might be measured in
            degrees.
         \item The actuator is the fire, which grows hotter when more air flows
            (when the flew angle corresponds to a more open state) and cooler
            when less air flows (when the flew angle corresponds to a more
            closed state).  The signal from the actuator is the rate of heat
            input to the water. This might be measured in Watts.
         \item The process is the water tank and the volume of the incubator,
            which is surrounded by the tank. The signal output by the process
            is the temperature measured by the thermometer. This signal might
            be a quantity measured in degrees Kelvin.
         \item The sensor is the thermometer (alcohol trapped by mercury),
            whose output signal is the height of the mercury where it touches
            the float.
         \item The input filter is the length of the riser, whose bottom is the
            float. The length of the riser corresponds to the flew's half-open
            setting when the alcohol has the desired temperature.  The
            reference signal is the height of the float when the flew is half
            open (or equivalently, when the desired temperature is obtained).
         \item The reference height for the float is compared with the actual
            height of the float by the flew linkage, which, in response to the
            change in the height of the float, moves away from the
            configuration in which the flew is half open. The displacement away
            from the reference position for the linkage is the error signal.
      \end{itemize}
      Drebbel's incubator provides an easy way to see why a high gain
      corresponds to a minimal error in the maintenance of the desired quantity
      by the control system. In my description of the incubator, I have
      supposed that the incubator is configured so that the flew is half open
      when the thermometer detects the ideal temperature. Suppose further,
      though, that the half-open position allows so much air flow as to raise
      the temperature. This will cause the alcohol in the thermometer to expand
      and to rotate the flew in the direction of the closed position. If
      stable, the control in this case will end up converging on a temperature
      at least a bit higher than the desired temperature. However, if the ratio
      of the change in the flew's angle to the change in temperature be large
      enough, then the controlled temperature will be only very slightly
      different from the desired temperature, regardless of which flew setting
      (half open, a quarter open, three-quarters open, etc.)~be chosen to
      correspond to the ideal temperature. This ratio is the loop gain.
   \item For the float-valve system, see Figure~\ref{fig:float-valve}.
      \begin{itemize}
         \item Functioning as the controller, the float, in response to its own
            offset from its reference level, allows water to flow from the
            supply through a certain cross-sectional area.  The offset is the
            error signal: As the float's level descends, the area increases; as
            the level ascends, the area decreases. The area is the controller's
            output signal; it might be measured in square centimeters.
         \item The actuator is the water-supply pipe, which has a certain water
            pressure. The actuator's input signal is the area across which the
            supply water may flow into the tank.  The actuator's output signal
            is the tank's inflow rate, which is a function of the supply's
            pressure and of the cross-sectional input area.  The actuator's
            output signal might be measured in cubic centimeters per second.
         \item The process is the water tank. The signal output by the process
            is the level of the water in the tank. The level might be measured
            in centimeters.
         \item The sensor is the float, whose output signal is its level, which
            might be measured in centimeters.
         \item The input filter is the level of the bottom terminus of the
            supply pipe. The level of the terminus determines the reference
            signal, which is the water level at which the float allows the
            supply to provide new water at half of the maximum rate (or
            equivalently, at which the desired water level is obtained).
         \item The reference float level is compared with the actual float
            level by the float itself, whose own offset from the reference
            level is the error signal. The error signal might be measured in
            centimeters.
      \end{itemize}
      The float-valve system, like Drebbel's incubator, allows one easily to
      see how a high loop gain minimizes the error in the controlled variable.
      I have, as for the case of the incubator, supposed that the reference
      value (which is compared to the sensor's measurement) corresponds to the
      state in which the actuator is working at half of its capacity. As
      before, this might well cause the controlled variable to assume a value
      different from what is desired (for example, if the rate of outflow from
      the tank were zero). However, a sufficiently high gain---in this case, a
      large change in inflow rate for a small change in water level---would
      reduce the error in the controlled variable to an arbitrarily small
      value.
   \item For Watt's steam engine with fly-ball governor, see
      Figure~\ref{fig:steam-engine}.
      \begin{itemize}
         \item Functioning as the controller, the valve, in response to the
            ball's angular offset from its reference position, shifts away from
            the half-open position. The offset is the error signal. The control
            signal is the angle of the valve. The angle of the valve might be
            measured in degrees.
         \item The actuator is the steam supply. The actuator's input signal is
            the valve's angle. The actuator's output signal is the steam's flow
            rate. The flow rate might be measured in cubic centimeters per
            second.
         \item The process is the engine. The signal output by the engine is
            the engine's spin rate. The spin rate might be measured in
            revolutions per second.
         \item The sensor is the ball assembly, whose output signal is the
            angle that each ball makes with the vertical direction. This angle
            might be measured in degrees.
         \item The input filter is the valve linkage, which is configured so
            that the valve is half open at the desired spin rate.
         \item The reference ball angle is compared with the measured ball
            angle by the ball assembly itself, which translates the error
            signal to the valve. The error signal might be measured in degrees,
            and there is some gain between the angular offset of the ball
            assembly and the angular offset of the valve.
      \end{itemize}
      As in the previous two examples, a high loop gain will allow the control
      system to keep the spin rate near the desired rate, regardless of what
      valve angle is chosen to correspond to the desired rate.
\end{enumerate}
\begin{figure}
   \begin{center}
      \begin{small}
         \input{pr01-01-c}
      \end{small}
      \caption{Block diagram for float-controller of water level.}
      \label{fig:float-valve}
   \end{center}
\end{figure}
\begin{figure}
   \begin{center}
      \begin{small}
         \input{pr01-01-d}
      \end{small}
      \caption{Block diagram for steam engine with governor.}
      \label{fig:steam-engine}
   \end{center}
\end{figure}

\subsection{Thermostat}

\subsubsection{Problem}

Identify the physical principles and describe the operation of the thermostat
in your home or office.

\subsubsection{Solution}

The thermostat in my house is a digital device that incorporates a temperature
sensor, an adjustable controller, and electrical switches that control the
furnace, the refrigeration system, and the blower. The precise nature of the
temperature sensor is not evident, but it probably produces a voltage that is
proportional to the temperature in the room where the thermometer sits. An
analog-to-digital converter transforms the voltage into an integer number that
the digital controller can compare to the adjustable set-point. There is a
dead-band of measured temperature outside of which the controller enables the
actuator until the measured temperature crosses back into the deadband, through
the set point at its center, and across to the temperature on the opposite side
of the deadband. In cooling mode, the controller enables the refrigeration
system and the blower when the measured temperature exits the high side of the
deadband. In heating mode, the controller enables the furnace and the blower
when the measured temperature exits the low side of the dead band.

