
\documentclass{book}

\usepackage{amsmath,amssymb}
\usepackage{amsthm}
\usepackage{thmtools}

% Setup of thmtools. See
% 'http://tex.stackexchange.com/questions/19947/example-solution-environment',
% whose approach is modified so that 'problem' replaces 'example'.
\declaretheoremstyle[
   spaceabove=6pt, spacebelow=6pt,
   headfont=\normalfont\bfseries,
   notefont=\mdseries, notebraces={(}{)},
   bodyfont=\normalfont,
   postheadspace=1em,
   numberwithin=section
]{prbstyle}
\declaretheoremstyle[
   spaceabove=6pt, spacebelow=6pt,
   headfont=\normalfont\bfseries,
   notefont=\mdseries, notebraces={(}{)},
   bodyfont=\normalfont,
   postheadspace=1em,
   headpunct={},
   qed=$\blacktriangleleft$,
   numbered=no
]{solstyle}
\declaretheorem[style=prbstyle]{problem}
\declaretheorem[style=solstyle]{solution}
% At this point, we have new environments, called 'problem' and 'solution'.

\author{Thomas E. Vaughan}

\title{Solutions to Problems in the Sixth Edition of {\it Feeback Control of
Dynamic Systems} (2010)}

\begin{document}

\maketitle


\chapter{An Overview and Brief History of Feedback Control}

\section{Review Questions}

\begin{problem}
   What are the main components of a feedback control system?
\end{problem}

\begin{figure}
   \begin{center}
      %\includegraphics{control}
      \begin{tikzpicture}
      \tikzset{vertex/.style = {shape=circle,draw,minimum size=10ex}}
      \tikzset{edge/.style = {->,> = latex'}}
      \fill[blue!40!white] (-1,-11) rectangle(1,-6.5);
      \node[blue,left] at (-1,-8.75) {plant};
      % vertices
      \node[vertex] (a) at  (0,  0) {input filter};
      \node[vertex] (b) at  (0, -2.5) {$\Sigma$};
      \node[above,outer sep=28pt] at (0.2, -2.5) {$+$};
      \node[below right,outer sep=12pt] at (0.4, -2.5) {$-$};
      \node[vertex] (c) at  (0, -5) {controller};
      \node[vertex] (d) at  (0, -7.5) {actuator};
      \node[vertex] (e) at  (0,-10) {process};
      \node[vertex] (f) at  (2.5,-10) {sensor};
      %edges
      \draw[edge] (a) to (b);
      \draw[edge] (b) to (c);
      \draw[edge] (c) to (d);
      \draw[edge] (d) to (e);
      \draw[edge] (e) to (f);
      \draw[edge] (f) to[bend right] (b);
      \end{tikzpicture}
      \label{fig:control}
      \caption{%
         Main components of a feedback-control system. The comparator is
         represented by the symbol `$\Sigma$', the symbol for summation,
         because a difference is a special kind of sum. The comparator takes a
         difference between two quantities by adding one to the negative of the
         other. In this case, the comparator adds the value from the input
         filter to the negative of the output of the sensor.%
      }
   \end{center}
\end{figure}

\begin{solution}
   A feedback control system has
   \begin{enumerate}
      \item a \emph{sensor} that measures the output that is to be controlled,
      \item an \emph{input filter} that provides a reference value against
         which the sensor's output is compared,
      \item a \emph{comparator} that computes the difference between the output
         of the sensor and the output of the input filter,
      \item a \emph{controller} that translates the output of the comparator
         into a control signal,
      \item a \emph{plant}, which consists of
         \begin{enumerate}
            \item an \emph{actuator} that receives the control signal and
               causes a physical change in the system and
            \item a \emph{process} that can be disturbed by environmental
               forces but is primarily influenced by the actuator to produce an
               output measured by the sensor.
         \end{enumerate}
   \end{enumerate}
\end{solution}




\end{document}

